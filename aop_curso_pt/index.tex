% Options for packages loaded elsewhere
\PassOptionsToPackage{unicode}{hyperref}
\PassOptionsToPackage{hyphens}{url}
\PassOptionsToPackage{dvipsnames,svgnames,x11names}{xcolor}
%
\documentclass[
  letterpaper,
  DIV=11,
  numbers=noendperiod]{scrreprt}

\usepackage{amsmath,amssymb}
\usepackage{lmodern}
\usepackage{iftex}
\ifPDFTeX
  \usepackage[T1]{fontenc}
  \usepackage[utf8]{inputenc}
  \usepackage{textcomp} % provide euro and other symbols
\else % if luatex or xetex
  \usepackage{unicode-math}
  \defaultfontfeatures{Scale=MatchLowercase}
  \defaultfontfeatures[\rmfamily]{Ligatures=TeX,Scale=1}
\fi
% Use upquote if available, for straight quotes in verbatim environments
\IfFileExists{upquote.sty}{\usepackage{upquote}}{}
\IfFileExists{microtype.sty}{% use microtype if available
  \usepackage[]{microtype}
  \UseMicrotypeSet[protrusion]{basicmath} % disable protrusion for tt fonts
}{}
\makeatletter
\@ifundefined{KOMAClassName}{% if non-KOMA class
  \IfFileExists{parskip.sty}{%
    \usepackage{parskip}
  }{% else
    \setlength{\parindent}{0pt}
    \setlength{\parskip}{6pt plus 2pt minus 1pt}}
}{% if KOMA class
  \KOMAoptions{parskip=half}}
\makeatother
\usepackage{xcolor}
\setlength{\emergencystretch}{3em} % prevent overfull lines
\setcounter{secnumdepth}{5}
% Make \paragraph and \subparagraph free-standing
\ifx\paragraph\undefined\else
  \let\oldparagraph\paragraph
  \renewcommand{\paragraph}[1]{\oldparagraph{#1}\mbox{}}
\fi
\ifx\subparagraph\undefined\else
  \let\oldsubparagraph\subparagraph
  \renewcommand{\subparagraph}[1]{\oldsubparagraph{#1}\mbox{}}
\fi

\usepackage{color}
\usepackage{fancyvrb}
\newcommand{\VerbBar}{|}
\newcommand{\VERB}{\Verb[commandchars=\\\{\}]}
\DefineVerbatimEnvironment{Highlighting}{Verbatim}{commandchars=\\\{\}}
% Add ',fontsize=\small' for more characters per line
\usepackage{framed}
\definecolor{shadecolor}{RGB}{241,243,245}
\newenvironment{Shaded}{\begin{snugshade}}{\end{snugshade}}
\newcommand{\AlertTok}[1]{\textcolor[rgb]{0.68,0.00,0.00}{#1}}
\newcommand{\AnnotationTok}[1]{\textcolor[rgb]{0.37,0.37,0.37}{#1}}
\newcommand{\AttributeTok}[1]{\textcolor[rgb]{0.40,0.45,0.13}{#1}}
\newcommand{\BaseNTok}[1]{\textcolor[rgb]{0.68,0.00,0.00}{#1}}
\newcommand{\BuiltInTok}[1]{\textcolor[rgb]{0.00,0.23,0.31}{#1}}
\newcommand{\CharTok}[1]{\textcolor[rgb]{0.13,0.47,0.30}{#1}}
\newcommand{\CommentTok}[1]{\textcolor[rgb]{0.37,0.37,0.37}{#1}}
\newcommand{\CommentVarTok}[1]{\textcolor[rgb]{0.37,0.37,0.37}{\textit{#1}}}
\newcommand{\ConstantTok}[1]{\textcolor[rgb]{0.56,0.35,0.01}{#1}}
\newcommand{\ControlFlowTok}[1]{\textcolor[rgb]{0.00,0.23,0.31}{#1}}
\newcommand{\DataTypeTok}[1]{\textcolor[rgb]{0.68,0.00,0.00}{#1}}
\newcommand{\DecValTok}[1]{\textcolor[rgb]{0.68,0.00,0.00}{#1}}
\newcommand{\DocumentationTok}[1]{\textcolor[rgb]{0.37,0.37,0.37}{\textit{#1}}}
\newcommand{\ErrorTok}[1]{\textcolor[rgb]{0.68,0.00,0.00}{#1}}
\newcommand{\ExtensionTok}[1]{\textcolor[rgb]{0.00,0.23,0.31}{#1}}
\newcommand{\FloatTok}[1]{\textcolor[rgb]{0.68,0.00,0.00}{#1}}
\newcommand{\FunctionTok}[1]{\textcolor[rgb]{0.28,0.35,0.67}{#1}}
\newcommand{\ImportTok}[1]{\textcolor[rgb]{0.00,0.46,0.62}{#1}}
\newcommand{\InformationTok}[1]{\textcolor[rgb]{0.37,0.37,0.37}{#1}}
\newcommand{\KeywordTok}[1]{\textcolor[rgb]{0.00,0.23,0.31}{#1}}
\newcommand{\NormalTok}[1]{\textcolor[rgb]{0.00,0.23,0.31}{#1}}
\newcommand{\OperatorTok}[1]{\textcolor[rgb]{0.37,0.37,0.37}{#1}}
\newcommand{\OtherTok}[1]{\textcolor[rgb]{0.00,0.23,0.31}{#1}}
\newcommand{\PreprocessorTok}[1]{\textcolor[rgb]{0.68,0.00,0.00}{#1}}
\newcommand{\RegionMarkerTok}[1]{\textcolor[rgb]{0.00,0.23,0.31}{#1}}
\newcommand{\SpecialCharTok}[1]{\textcolor[rgb]{0.37,0.37,0.37}{#1}}
\newcommand{\SpecialStringTok}[1]{\textcolor[rgb]{0.13,0.47,0.30}{#1}}
\newcommand{\StringTok}[1]{\textcolor[rgb]{0.13,0.47,0.30}{#1}}
\newcommand{\VariableTok}[1]{\textcolor[rgb]{0.07,0.07,0.07}{#1}}
\newcommand{\VerbatimStringTok}[1]{\textcolor[rgb]{0.13,0.47,0.30}{#1}}
\newcommand{\WarningTok}[1]{\textcolor[rgb]{0.37,0.37,0.37}{\textit{#1}}}

\providecommand{\tightlist}{%
  \setlength{\itemsep}{0pt}\setlength{\parskip}{0pt}}\usepackage{longtable,booktabs,array}
\usepackage{calc} % for calculating minipage widths
% Correct order of tables after \paragraph or \subparagraph
\usepackage{etoolbox}
\makeatletter
\patchcmd\longtable{\par}{\if@noskipsec\mbox{}\fi\par}{}{}
\makeatother
% Allow footnotes in longtable head/foot
\IfFileExists{footnotehyper.sty}{\usepackage{footnotehyper}}{\usepackage{footnote}}
\makesavenoteenv{longtable}
\usepackage{graphicx}
\makeatletter
\def\maxwidth{\ifdim\Gin@nat@width>\linewidth\linewidth\else\Gin@nat@width\fi}
\def\maxheight{\ifdim\Gin@nat@height>\textheight\textheight\else\Gin@nat@height\fi}
\makeatother
% Scale images if necessary, so that they will not overflow the page
% margins by default, and it is still possible to overwrite the defaults
% using explicit options in \includegraphics[width, height, ...]{}
\setkeys{Gin}{width=\maxwidth,height=\maxheight,keepaspectratio}
% Set default figure placement to htbp
\makeatletter
\def\fps@figure{htbp}
\makeatother
\newlength{\cslhangindent}
\setlength{\cslhangindent}{1.5em}
\newlength{\csllabelwidth}
\setlength{\csllabelwidth}{3em}
\newlength{\cslentryspacingunit} % times entry-spacing
\setlength{\cslentryspacingunit}{\parskip}
\newenvironment{CSLReferences}[2] % #1 hanging-ident, #2 entry spacing
 {% don't indent paragraphs
  \setlength{\parindent}{0pt}
  % turn on hanging indent if param 1 is 1
  \ifodd #1
  \let\oldpar\par
  \def\par{\hangindent=\cslhangindent\oldpar}
  \fi
  % set entry spacing
  \setlength{\parskip}{#2\cslentryspacingunit}
 }%
 {}
\usepackage{calc}
\newcommand{\CSLBlock}[1]{#1\hfill\break}
\newcommand{\CSLLeftMargin}[1]{\parbox[t]{\csllabelwidth}{#1}}
\newcommand{\CSLRightInline}[1]{\parbox[t]{\linewidth - \csllabelwidth}{#1}\break}
\newcommand{\CSLIndent}[1]{\hspace{\cslhangindent}#1}

\KOMAoption{captions}{tableheading}
\makeatletter
\@ifpackageloaded{tcolorbox}{}{\usepackage[many]{tcolorbox}}
\@ifpackageloaded{fontawesome5}{}{\usepackage{fontawesome5}}
\definecolor{quarto-callout-color}{HTML}{909090}
\definecolor{quarto-callout-note-color}{HTML}{0758E5}
\definecolor{quarto-callout-important-color}{HTML}{CC1914}
\definecolor{quarto-callout-warning-color}{HTML}{EB9113}
\definecolor{quarto-callout-tip-color}{HTML}{00A047}
\definecolor{quarto-callout-caution-color}{HTML}{FC5300}
\definecolor{quarto-callout-color-frame}{HTML}{acacac}
\definecolor{quarto-callout-note-color-frame}{HTML}{4582ec}
\definecolor{quarto-callout-important-color-frame}{HTML}{d9534f}
\definecolor{quarto-callout-warning-color-frame}{HTML}{f0ad4e}
\definecolor{quarto-callout-tip-color-frame}{HTML}{02b875}
\definecolor{quarto-callout-caution-color-frame}{HTML}{fd7e14}
\makeatother
\makeatletter
\makeatother
\makeatletter
\@ifpackageloaded{caption}{}{\usepackage{caption}}
\AtBeginDocument{%
\ifdefined\contentsname
  \renewcommand*\contentsname{Table of contents}
\else
  \newcommand\contentsname{Table of contents}
\fi
\ifdefined\listfigurename
  \renewcommand*\listfigurename{List of Figures}
\else
  \newcommand\listfigurename{List of Figures}
\fi
\ifdefined\listtablename
  \renewcommand*\listtablename{List of Tables}
\else
  \newcommand\listtablename{List of Tables}
\fi
\ifdefined\figurename
  \renewcommand*\figurename{Figure}
\else
  \newcommand\figurename{Figure}
\fi
\ifdefined\tablename
  \renewcommand*\tablename{Table}
\else
  \newcommand\tablename{Table}
\fi
}
\@ifpackageloaded{float}{}{\usepackage{float}}
\floatstyle{ruled}
\@ifundefined{c@chapter}{\newfloat{codelisting}{h}{lop}}{\newfloat{codelisting}{h}{lop}[chapter]}
\floatname{codelisting}{Listing}
\newcommand*\listoflistings{\listof{codelisting}{List of Listings}}
\makeatother
\makeatletter
\@ifpackageloaded{caption}{}{\usepackage{caption}}
\@ifpackageloaded{subcaption}{}{\usepackage{subcaption}}
\makeatother
\makeatletter
\@ifpackageloaded{tcolorbox}{}{\usepackage[many]{tcolorbox}}
\makeatother
\makeatletter
\@ifundefined{shadecolor}{\definecolor{shadecolor}{rgb}{.97, .97, .97}}
\makeatother
\makeatletter
\makeatother
\ifLuaTeX
  \usepackage{selnolig}  % disable illegal ligatures
\fi
\IfFileExists{bookmark.sty}{\usepackage{bookmark}}{\usepackage{hyperref}}
\IfFileExists{xurl.sty}{\usepackage{xurl}}{} % add URL line breaks if available
\urlstyle{same} % disable monospaced font for URLs
\hypersetup{
  pdftitle={Acessibilidade urbana e avaliação de impacto},
  pdfauthor={Rafael Pererira, \ldots{}},
  colorlinks=true,
  linkcolor={blue},
  filecolor={Maroon},
  citecolor={Blue},
  urlcolor={Blue},
  pdfcreator={LaTeX via pandoc}}

\title{Acessibilidade urbana e avaliação de impacto}
\author{Rafael Pererira, \ldots{}}
\date{5/21/2022}

\begin{document}
\maketitle
\ifdefined\Shaded\renewenvironment{Shaded}{\begin{tcolorbox}[interior hidden, borderline west={3pt}{0pt}{shadecolor}, enhanced, boxrule=0pt, breakable, sharp corners, frame hidden]}{\end{tcolorbox}}\fi

\renewcommand*\contentsname{Table of contents}
{
\hypersetup{linkcolor=}
\setcounter{tocdepth}{2}
\tableofcontents
}
\hypertarget{apresentauxe7uxe3o}{%
\section*{Apresentação}\label{apresentauxe7uxe3o}}
\addcontentsline{toc}{section}{Apresentação}

Esse site /livro/ tem como objetivo curso equipar gestores públicos,
analistas e pesquisadores de planejamento e transporte urbano com o
conhecimento e as habilidades práticas para fazer estudos e avaliações
de impacto sobre acessibilidade urbana. Além de uma visão geral sobre
conceitos e indicadores de acessibilidade, o curso ensina como é
possível analisar dados espaciais e de redes de transporte em \texttt{R}
para calcular e visualizar o acesso a oportunidades nas cidades
brasileiras. O curso tem caráter prático `mão na masssa', e é baseado em
exemplos e exercícios reproduzíveis usando R.

O minicurso é elaborado pelo Ipea no âmbito da parceria entre Ipea e a
Secretaria de Mobilidade e Desenvolvimento Regional e Urbano (SMDRU) do
Ministério do Desenvolvimento Regional (MDR).

\begin{tcolorbox}[enhanced jigsaw, colframe=quarto-callout-note-color-frame, opacityback=0, toprule=.15mm, breakable, left=2mm, colback=white, arc=.35mm, rightrule=.15mm, bottomrule=.15mm, leftrule=.75mm]
\begin{minipage}[t]{5.5mm}
\textcolor{quarto-callout-note-color}{\faInfo}
\end{minipage}%
\begin{minipage}[t]{\textwidth - 5.5mm}
Esse livro / site foi escrito usando
\href{https://www.r-project.org/}{r} e
\href{https://quarto.org}{Quarto}. O código utilizado para criar o curso
pode ser encontrado
\href{https://github.com/ipeaGIT/aop_curso/tree/main/aop_curso_pt}{neste
repositório}.\end{minipage}%
\end{tcolorbox}

\hypertarget{licensa-de-uso}{%
\section*{Licensa de uso}\label{licensa-de-uso}}
\addcontentsline{toc}{section}{Licensa de uso}

definir licença

\part{PARTE 1: Introdução a acessibilidade urbana}

\textbf{Objetivo}: O objetivo desse cap�tulo � 1. Apresentar o que o
conceito de acessibilidade urbana e esclarecer a diferen�a entre
acessibilidade e mobilidade 2. Apresentar uma visão geral sobre os
principais e indicadores para medir acessibilidade.

\hypertarget{o-que-uxe9-acessibilidade}{%
\chapter{O que é acessibilidade?}\label{o-que-uxe9-acessibilidade}}

\hypertarget{definiuxe7uxe3o-de-acessibilidade-urbana}{%
\section{Definição de acessibilidade
urbana}\label{definiuxe7uxe3o-de-acessibilidade-urbana}}

tres componentes:

- Infraestrutura

- Uso do solo

- Pessoas

\hypertarget{diferenuxe7a-entre-micro-e-macro-acessibilidade}{%
\section{Diferença entre micro e macro
acessibilidade}\label{diferenuxe7a-entre-micro-e-macro-acessibilidade}}

\hypertarget{diferenuxe7a-entre-acessibilidade-e-mobilidade-urbana}{%
\section{Diferença entre acessibilidade e mobilidade
urbana}\label{diferenuxe7a-entre-acessibilidade-e-mobilidade-urbana}}

\hypertarget{indicadores-de-acessibilidade}{%
\chapter{Indicadores de
acessibilidade}\label{indicadores-de-acessibilidade}}

\hypertarget{medidas-baseadas-em-lugares}{%
\section{Medidas baseadas em
lugares}\label{medidas-baseadas-em-lugares}}

\hypertarget{muxednima-distuxe2ncia-ou-tempo-de-viagem}{%
\subsection{1.3.1 Mínima distância ou tempo de
viagem}\label{muxednima-distuxe2ncia-ou-tempo-de-viagem}}

\hypertarget{medida-cumulativa-de-oportunidades}{%
\subsection{1.3.2 Medida cumulativa de
oportunidades}\label{medida-cumulativa-de-oportunidades}}

\hypertarget{medidas-gravitacionais}{%
\subsection{1.3.3 Medidas gravitacionais}\label{medidas-gravitacionais}}

\hypertarget{indicadores-com-competiuxe7uxe3o-floating-catchment-area}{%
\subsection{1.3.4 Indicadores com competição: Floating catchment
area}\label{indicadores-com-competiuxe7uxe3o-floating-catchment-area}}

\hypertarget{medidas-baseadas-em-pessoas}{%
\section{Medidas baseadas em
pessoas}\label{medidas-baseadas-em-pessoas}}

\part{PARTE 2: Dados do Projeto AOP}

\textbf{Objetivo}: o objetivo deste capítulo é mostrar como você pode
usar o \texttt{R} para fazer download dos dados do projeto Acesso a
Oportunidades (AOP) utilizando o pacote \texttt{aopdata}.

Esses dados permitem você analisar diversas estimativas de de
acessibilidade a empregos e serviços públicos, como escolas,
estabelecimentos de saúde e centros de referência para assistência
social (CRAS). Essas estimativas são calculadas usando vários
indicadores de acessibilidade, considerando diferentes modos de
transporte (caminhada, bicicleta, transporte público e automóvel),
diferentes horários do dia (pico e fora-pico) e diferentes grupos
populacionais (segundo níveis de renda, rara, sexo e idade).

Nesta versão, a base de dados traz essas informações para os anos de
2017, 2018 e 2019, se apoiando em uma única metodologia consistente para
as 20 maiores cidades do Brasil. As metodologias utilizadas para gerar
estes dados são apresentadas em detalhe em publicações separadas, para
os dados populacionais e de uso do solo (ref), e para os dados de
acessibilidade (ref).

\hypertarget{dados-de-populauxe7uxe3o-e-socioeconuxf4micos}{%
\chapter{Dados de população e
socioeconômicos}\label{dados-de-populauxe7uxe3o-e-socioeconuxf4micos}}

\hypertarget{download-dos-dados}{%
\section{Download dos dados}\label{download-dos-dados}}

\hypertarget{mapa-de-populauxe7uxe3o-total}{%
\section{Mapa de população total}\label{mapa-de-populauxe7uxe3o-total}}

\hypertarget{mapa-de-populauxe7uxe3o-por-renda}{%
\section{Mapa de população por
renda}\label{mapa-de-populauxe7uxe3o-por-renda}}

\hypertarget{mapa-de-populauxe7uxe3o-por-cor}{%
\section{Mapa de população por
cor}\label{mapa-de-populauxe7uxe3o-por-cor}}

\begin{Shaded}
\begin{Highlighting}[]
\DecValTok{1} \SpecialCharTok{+} \DecValTok{1}
\end{Highlighting}
\end{Shaded}

\begin{verbatim}
[1] 2
\end{verbatim}

\hypertarget{dados-de-distribuiuxe7uxe3o-espacial-de-oportunidades}{%
\chapter{Dados de distribuição espacial de
oportunidades}\label{dados-de-distribuiuxe7uxe3o-espacial-de-oportunidades}}

\hypertarget{download-dos-dados-1}{%
\section{Download dos dados}\label{download-dos-dados-1}}

\hypertarget{mapa-de-empregos}{%
\section{Mapa de empregos}\label{mapa-de-empregos}}

\hypertarget{mapa-de-escolas}{%
\section{Mapa de escolas}\label{mapa-de-escolas}}

\hypertarget{mapa-de-servios-de-sade}{%
\section{Mapa de servi�os de sa�de}\label{mapa-de-servios-de-sade}}

\hypertarget{mapa-de-cras}{%
\section{Mapa de CRAS}\label{mapa-de-cras}}

\begin{Shaded}
\begin{Highlighting}[]
\DecValTok{1} \SpecialCharTok{+} \DecValTok{1}
\end{Highlighting}
\end{Shaded}

\begin{verbatim}
[1] 2
\end{verbatim}

\hypertarget{estimativas-e-mapas-de-acessibilidade}{%
\chapter{Estimativas e mapas de
acessibilidade}\label{estimativas-e-mapas-de-acessibilidade}}

\hypertarget{download-dos-dados-2}{%
\section{Download dos dados}\label{download-dos-dados-2}}

\hypertarget{estimativas-de-acessibilidade}{%
\section{Estimativas de
acessibilidade}\label{estimativas-de-acessibilidade}}

ilustrar alguns indicadores para pegar a logica do dicionario de dados

\hypertarget{geografia-do-acesso-a-oportunidades}{%
\section{Geografia do acesso a
oportunidades}\label{geografia-do-acesso-a-oportunidades}}

a

\hypertarget{mapa-tmi-a-sade}{%
\subsection{Mapa TMI a sa�de}\label{mapa-tmi-a-sade}}

a

\hypertarget{mapa-cma-empregos}{%
\subsection{Mapa CMA empregos}\label{mapa-cma-empregos}}

a

\hypertarget{desigualdades-de-acesso-a-oportunidades}{%
\section{Desigualdades de acesso a
oportunidades}\label{desigualdades-de-acesso-a-oportunidades}}

\begin{Shaded}
\begin{Highlighting}[]
\DecValTok{1} \SpecialCharTok{+} \DecValTok{1}
\end{Highlighting}
\end{Shaded}

\begin{verbatim}
[1] 2
\end{verbatim}

\part{PARTE 3: Dados de transporte público GTFS}

\hypertarget{dados-de-transporte-puxfablico-em-formato-gtfs}{%
\chapter{Dados de transporte público em formato
GTFS}\label{dados-de-transporte-puxfablico-em-formato-gtfs}}

\textbf{Objetivo}: apresentar uma visão geral sobre o que são dados
GTFS; onde conseguir dados de GTFS no Brasil; mostrar como fazer algumas
análises básicas e edições dos dados de GTFS usando o pacote gtfstools

\hypertarget{o-que-uxe9-gtfs}{%
\section{O que é GTFS}\label{o-que-uxe9-gtfs}}

\hypertarget{estrutura-dos-arquivos-de-gtfs}{%
\section{Estrutura dos arquivos de
GTFS}\label{estrutura-dos-arquivos-de-gtfs}}

\hypertarget{onde-encontrar-gtfs-de-cidades-brasileiras}{%
\section{Onde encontrar GTFS de cidades
brasileiras}\label{onde-encontrar-gtfs-de-cidades-brasileiras}}

\hypertarget{como-extrair-anuxe1lises-buxe1sicas-de-um-gtfs-pacote-gtfstools}{%
\section{Como extrair análises básicas de um GTFS (pacote
gtfstools)}\label{como-extrair-anuxe1lises-buxe1sicas-de-um-gtfs-pacote-gtfstools}}

\hypertarget{cuxe1lculo-de-velocidade-das-linhas}{%
\section{Cálculo de velocidade das
linhas}\label{cuxe1lculo-de-velocidade-das-linhas}}

\hypertarget{cuxe1lculo-de-frequuxeancia-das-linhas}{%
\section{Cálculo de frequência das
linhas}\label{cuxe1lculo-de-frequuxeancia-das-linhas}}

\hypertarget{mapear-a-rede-de-transporte-puxfablico}{%
\section{Mapear a rede de transporte
público}\label{mapear-a-rede-de-transporte-puxfablico}}

\hypertarget{como-fazer-ediuxe7uxf5es-na-rede-de-transporte-puxfablico-pacote-gtfstools}{%
\section{Como fazer edições na rede de transporte público (pacote
gtfstools)}\label{como-fazer-ediuxe7uxf5es-na-rede-de-transporte-puxfablico-pacote-gtfstools}}

\part{PARTE 4: Calculando acessibilidade}

\hypertarget{calculando-acessibilidade-urbana-com-r5r}{%
\chapter{\texorpdfstring{Calculando acessibilidade urbana com
\texttt{r5r}}{Calculando acessibilidade urbana com r5r}}\label{calculando-acessibilidade-urbana-com-r5r}}

Objetivo: mostrar como calcular acessibilidade urbana usando o pacote
r5r

6.1 Função `accessibility\{r5r\}`, diferentes indicadores

6.2 Mapa de acessibilidade

\begin{Shaded}
\begin{Highlighting}[]
\DecValTok{1} \SpecialCharTok{+} \DecValTok{1}
\end{Highlighting}
\end{Shaded}

\begin{verbatim}
[1] 2
\end{verbatim}

\part{PARTE 5: Avaliação de impacto}

\hypertarget{comparando-a-acessibilidade-entre-dois-cenuxe1rios-de-transporte}{%
\chapter{Comparando a acessibilidade entre dois cenários de
transporte}\label{comparando-a-acessibilidade-entre-dois-cenuxe1rios-de-transporte}}

\textbf{Objetivo}: mostrar como avaliar o impacto de acessibilidade de
uma política que altera a frequência de algumas linhas de transporte

\hypertarget{alterar-frequuxeancia-de-gtfs}{%
\section{7.1 Alterar frequência de
GTFS}\label{alterar-frequuxeancia-de-gtfs}}

\hypertarget{calcular-acessibilidade-nos-cenuxe1rios-antes-e-depois}{%
\section{7.2 Calcular acessibilidade nos cenários antes e
depois}\label{calcular-acessibilidade-nos-cenuxe1rios-antes-e-depois}}

\hypertarget{mapa-do-impacto-de-acessibilidade}{%
\section{Mapa do impacto de
acessibilidade}\label{mapa-do-impacto-de-acessibilidade}}

\hypertarget{como-impacto-de-acessibilidade-se-distribui-entre-grupos-sociais}{%
\section{Como impacto de acessibilidade se distribui entre grupos
sociais}\label{como-impacto-de-acessibilidade-se-distribui-entre-grupos-sociais}}

\hypertarget{comparando-a-acessibilidade-entre-dois-cenuxe1rios-de-uso-do-solo}{%
\chapter{Comparando a acessibilidade entre dois cenários de uso do
solo}\label{comparando-a-acessibilidade-entre-dois-cenuxe1rios-de-uso-do-solo}}

\textbf{Objetivo}: mostrar como avaliar o impacto de acessibilidade de
uma política que (a) constrói nova escola, ou (b) aumenta densidade de
população em determinadas áreas da cidade

\hypertarget{simulauxe7uxe3o-de-aumentando-de-densidade-populacional}{%
\section{Simulação de aumentando de densidade
populacional}\label{simulauxe7uxe3o-de-aumentando-de-densidade-populacional}}

\hypertarget{calcular-acessibilidade-nos-cenuxe1rios-antes-e-depois-1}{%
\section{Calcular acessibilidade nos cenários antes e
depois}\label{calcular-acessibilidade-nos-cenuxe1rios-antes-e-depois-1}}

\hypertarget{mapa-do-impacto-de-acessibilidade-1}{%
\section{Mapa do impacto de
acessibilidade}\label{mapa-do-impacto-de-acessibilidade-1}}

\hypertarget{como-impacto-de-acessibilidade-se-distribui-entre-grupos-sociais-1}{%
\section{Como impacto de acessibilidade se distribui entre grupos
sociais}\label{como-impacto-de-acessibilidade-se-distribui-entre-grupos-sociais-1}}

\hypertarget{references}{%
\chapter*{References}\label{references}}
\addcontentsline{toc}{chapter}{References}

\hypertarget{refs}{}
\begin{CSLReferences}{0}{0}
\end{CSLReferences}

\appendix
\addcontentsline{toc}{part}{Appendices}

\hypertarget{nouxe7uxf5es-buxe1sicas-de-r}{%
\chapter{Noções básicas de R}\label{nouxe7uxf5es-buxe1sicas-de-r}}

\hypertarget{objetos}{%
\section{Objetos}\label{objetos}}

\begin{Shaded}
\begin{Highlighting}[]
\NormalTok{a }\OtherTok{\textless{}{-}} \DecValTok{1}

\NormalTok{a }\SpecialCharTok{+} \DecValTok{1}
\end{Highlighting}
\end{Shaded}

\begin{verbatim}
[1] 2
\end{verbatim}

\begin{Shaded}
\begin{Highlighting}[]
\NormalTok{a }\OtherTok{\textless{}{-}} \FunctionTok{c}\NormalTok{(}\DecValTok{1}\NormalTok{,}\DecValTok{2}\NormalTok{,}\DecValTok{3}\NormalTok{)}

\NormalTok{a }\SpecialCharTok{+} \DecValTok{1}
\end{Highlighting}
\end{Shaded}

\begin{verbatim}
[1] 2 3 4
\end{verbatim}

Texto

\begin{Shaded}
\begin{Highlighting}[]
\NormalTok{a }\OtherTok{\textless{}{-}} \StringTok{"Bom dia"}

\FunctionTok{paste}\NormalTok{(a, }\StringTok{\textquotesingle{}Joana\textquotesingle{}}\NormalTok{)}
\end{Highlighting}
\end{Shaded}

\begin{verbatim}
[1] "Bom dia Joana"
\end{verbatim}

\hypertarget{data.frames}{%
\section{Data.frames}\label{data.frames}}

\hypertarget{como-importar-e-exportar-arquivos}{%
\section{Como importar e exportar
arquivos}\label{como-importar-e-exportar-arquivos}}

\hypertarget{funuxe7uxf5es}{%
\section{Funções}\label{funuxe7uxf5es}}

\hypertarget{visualizauxe7uxe3o-de-dados-com-ggplot2}{%
\section{\texorpdfstring{Visualização de dados com
\texttt{ggplot2}}{Visualização de dados com ggplot2}}\label{visualizauxe7uxe3o-de-dados-com-ggplot2}}

```



\end{document}
